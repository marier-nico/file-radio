\begin{longtable}{|l|p{0.8\textwidth}|}
\multicolumn{2}{r}{Suite à la page suivante} \\
\endfoot

\multicolumn{2}{r}{} \\
\endlastfoot

\hline
    \rowcolor{Gray}
    \multicolumn{2}{|l|}{1} \\
\hline
    Acteur ou rôle & Physicien  \\
\hline
    Scénario ou story & En tant que physicien, je veux me renseigner sur les antennes et la télécommunication \\
\hline
    Détail et description &
        \begin{enumerate}[label*=\arabic*.]
       \item \colorbox{Green}{\parbox{13cm}{Recherche d'informations sur Internet et au département de télécommunication au Cégep.}}
            \begin{enumerate}[label*=\arabic*.]
                    \item Qui et temps :
                    \begin{enumerate}[label*=\arabic*.]
                        \item (TG-H) et (GR)
                        \item (1h)
                    \end{enumerate}
                    \item Préconditions :
                    \begin{enumerate}[label*=\arabic*.]
                    \item Avoir un circuit émetteur relativement fonctionnel.
                    \item Avoir un circuit récepteur relativement fonctionnel.
                    \end{enumerate}
                    \item Règles d’affaires :
                    \begin{enumerate}[label*=\arabic*.]
                        \item Recherche sur la concordance d'impédance dans un circuit doté d'une antenne.
                        \item Rencontre avec un professeur du département de télécommunication.
                    \end{enumerate}
                    \item Tests d'acceptation de cet item :
                    \begin{enumerate}[label*=\arabic*.]
                        \item Il n'y en pas vraiment.
                    \end{enumerate}
                    \item Post-conditions :
                    \begin{enumerate}[label*=\arabic*.]
                        \item Comprendre l'influence des antennes sur les circuits.
                    \end{enumerate}
                \end{enumerate}
        \end{enumerate} \\
\hline
    Tests d'acceptation & Vérifier nos recherches avec des profs de technique en télécommunications.\\

\hline
    Complexité & 1 \\
\hline
    Effort & 1 \\
\hline
    Commentaires & \\
    
\hline
    \rowcolor{Gray}
    \multicolumn{2}{|l|}{2} \\
\hline
    Acteur ou rôle & Physicien  \\
\hline
    Scénario ou story & En tant que physicien, je veux optimiser le circuit de l'émetteur et du récepteur. \\
\hline
    Détail et description &
        \begin{enumerate}[label*=\arabic*.]
            \item  \colorbox{Green}{\parbox{13cm}{ Accumuler le matériel nécessaire à la conception de meilleures antennes}}
            \begin{enumerate}[label*=\arabic*.]
                    \item Qui et temps :
                    \begin{enumerate}[label*=\arabic*.]
                        \item (TG-H) et (GR)
                        \item (30 minutes)
                    \end{enumerate}
                    \item Préconditions :
                    \begin{enumerate}[label*=\arabic*.]
                        \item Avoir un schéma de construction fini et la liste du matériel.
                    \end{enumerate}
                    \item Règles d’affaires :
                    \begin{enumerate}[label*=\arabic*.]
                        \item Doit respecter le matériel déterminé dans le schéma de construction.
                        %Yeet moi ca la le dessin thomas chou <3
                    \end{enumerate}
                    \item Tests d'acceptation de cet item :
                    \begin{enumerate}[label*=\arabic*.]
                        \item Vérifier si nous possédons bien chaque pièce nécessaire à la complétion des antennes.
                    \end{enumerate}
                    \item Post-conditions :
                    \begin{enumerate}[label*=\arabic*.]
                        \item Nous allons pouvoir procéder au montage des antennes.
                    \end{enumerate}
                \end{enumerate}
            \item  \colorbox{Green}{\parbox{13cm}{ Procéder à la conception de meilleures antennes}}
                \begin{enumerate}[label*=\arabic*.]
                    \item Qui et temps :
                    \begin{enumerate}[label*=\arabic*.]
                        \item (TG-H) et (GR)
                        \item (4h)
                    \end{enumerate}
                    \item Préconditions : 
                    \begin{enumerate}[label*=\arabic*.]
                        \item Avoir les pièces nécessaires à la conception en main.
                        \item Avoir le plan de conception achevé et vérifié.
                    \end{enumerate}
                    \item Règles d’affaires :
                    \begin{enumerate}[label*=\arabic*.]
                        \item Suivre le schéma de conception et construire les antennes
                    \end{enumerate}
                    \item Tests d'acceptation de cet item :
                    \begin{enumerate}[label*=\arabic*.]
                        \item Vérifier que l'antenne d'envoi transmet des ondes assez fortes.
                        \item Vérifier que l'antenne de récepteur reçoit assez de puissance.
                    \end{enumerate}
                    \item Post-conditions :
                    \begin{enumerate}[label*=\arabic*.]
                        \item L'utilisateur peut capter des ondes radio de manières efficaces.
                    \end{enumerate}
                \end{enumerate}
        \end{enumerate} \\
\hline
    Tests d'acceptation & Réception d'ondes pour en évaluer la puissance et l'exactitude.\\

\hline
    Complexité & 3 \\
\hline
    Effort & 6 \\
\hline
    Commentaires & \\

\hline
    \rowcolor{Gray}
    \multicolumn{2}{|l|}{3} \\
\hline
    Acteur ou rôle & Développeur \\
\hline
    Scénario ou story & En tant que développeur, je veux pouvoir animer les interfaces pour l'envoi et la réception. \\
\hline
    Détail et description &
        \begin{enumerate}[label*=\arabic*.]
            \item \colorbox{Green}{\parbox{10cm}{  Ajouter des animations sur les interfaces pour visualiser l'envoi}}\colorbox{Red}{\parbox{3cm}{ et la réception }} \colorbox{Green}{\parbox{5cm}{ des fichiers}}.
                \begin{enumerate}[label*=\arabic*.]
                                \item Qui et temps :
                                \begin{enumerate}[label*=\arabic*.]
                                    \item (CAD)
                                    \item (5h)
                                \end{enumerate}
                                \item Préconditions :
                                \begin{enumerate}[label*=\arabic*.]
                                    \item Avoir les vue FXML à jour.
                                    \item Avoir fait une recherche sur les animations JavaFx.
                                \end{enumerate}
                                \item Règles d'affaires :
                                \begin{enumerate}[label*=\arabic*.]
                                    \item Afficher une animation dans le haut des vues.
                                \end{enumerate}
                                \item Règles d'affaires alternatives :
                                \begin{enumerate}[label*=\arabic*.]
                                    \item Ne pas afficher d'animation.
                                \end{enumerate}
                                \item Tests d'acceptation de cet item :
                                \begin{enumerate}[label*=\arabic*.]
                                    \item Les tests seront au niveau visuel. S'il y a un problème d'affichage, on pourra le voir.
                                \end{enumerate}
                                \item Post-conditions :
                                \begin{enumerate}[label*=\arabic*.]
                                    \item Les interfaces devront pouvoir afficher une animation représentant l'envoi d'un fichier.
                                \end{enumerate}
                            \end{enumerate}
             \item  \colorbox{Green}{\parbox{13cm}{ Modifier le slider de fréquence pour qu'il devienne un slider qui gère la vitesse de l'envoi d'un bit et le binder aux autres informations.}}
                \begin{enumerate}[label*=\arabic*.]
                                \item Qui et temps :
                                \begin{enumerate}[label*=\arabic*.]
                                    \item (CAD)
                                    \item (2h)
                                \end{enumerate}
                                \item Préconditions :
                                \begin{enumerate}[label*=\arabic*.]
                                    \item Avoir la vue FXML à jour.
                                \end{enumerate}
                                \item Règles d'affaires :
                                \begin{enumerate}[label*=\arabic*.]
                                    \item Modifier la vitesse d'un bit avec la slider.
                                \end{enumerate}
                                \item Règles d'affaires alternatives :
                                \begin{enumerate}[label*=\arabic*.]
                                    \item Éliminer le slider, comme il deviendra désuet.
                                \end{enumerate}
                                \item Tests d'acceptation de cet item :
                                \begin{enumerate}[label*=\arabic*.]
                                    \item Les tests seront au niveau visuel. S'il y a un problème d'affichage, on pourra le voir.
                                \end{enumerate}
                                \item Post-conditions :
                                \begin{enumerate}[label*=\arabic*.]
                                    \item Le slider pourra modifier la vitesse d'envoi d'un bit.
                                \end{enumerate}
                            \end{enumerate}
        \end{enumerate} \\
\hline
    Tests d'acceptation & Les tests seront au niveau visuel. S'il y a un problème d'affichage, on pourra le voir. \\
\hline
    Complexité & 3 \\
\hline
    Effort & 3 \\
\hline
    Commentaires & Puisque nous ne savons pas encore comment va fonctionner la réception des informations, il est difficile de paufiner la vue de cette dernière et donc de la finaliser. \\
    
\hline
    \rowcolor{Gray}
    \multicolumn{2}{|l|}{4} \\
\hline
    Acteur ou rôle & Développeur \\
\hline
    Scénario ou story & En tant que développeur, je veux pouvoir optimiser les interfaces pour l'envoi et la réception. \\
\hline
    Détail et description &
        \begin{enumerate}[label*=\arabic*.]
            \item  \colorbox{Green}{\parbox{13cm}{ Modifier les dimensions des vues et les informations disponibles sur l'envoi du fichier.}}
                \begin{enumerate}[label*=\arabic*.]
                                \item Qui et temps :
                                \begin{enumerate}[label*=\arabic*.]
                                    \item (CAD)
                                    \item (3h)
                                \end{enumerate}
                                \item Préconditions :
                                \begin{enumerate}[label*=\arabic*.]
                                    \item Avoir la vue FXML à jour.
                                \end{enumerate}
                                \item Règles d'affaires :
                                \begin{enumerate}[label*=\arabic*.]
                                    \item Modifier les dimensions pour avoir des interfaces plus grandes.
                                    \item Modifier et ajouter des informations sur la vitesse et le son produit par l'envoi.
                                \end{enumerate}
                                \item Règles d'affaires alternatives :
                                \begin{enumerate}[label*=\arabic*.]
                                    \item Il n'y en a pas.
                                \end{enumerate}
                                \item Tests d'acceptation de cet item :
                                \begin{enumerate}[label*=\arabic*.]
                                    \item Les tests seront au niveau visuel. S'il y a un problème d'affichage, on pourra le voir.
                                \end{enumerate}
                                \item Post-conditions :
                                \begin{enumerate}[label*=\arabic*.]
                                    \item Les vues seront plus grandes et l'information disponible sera plus spécifique.
                                \end{enumerate}
                            \end{enumerate}
             \item  \colorbox{Green}{\parbox{13cm}{ Ajouter de la documentation aux classes des contrôleurs.}}
                \begin{enumerate}[label*=\arabic*.]
                                \item Qui et temps :
                                \begin{enumerate}[label*=\arabic*.]
                                    \item (CAD)
                                    \item (1h)
                                \end{enumerate}
                                \item Préconditions :
                                \begin{enumerate}[label*=\arabic*.]
                                    \item Avoir terminer les contrôleurs.
                                \end{enumerate}
                                \item Règles d'affaires :
                                \begin{enumerate}[label*=\arabic*.]
                                    \item Ajouter la JavaDoc
                                \end{enumerate}
                                \item Règles d'affaires alternatives :
                                \begin{enumerate}[label*=\arabic*.]
                                    \item Il n'y en a pas.
                                \end{enumerate}
                                \item Tests d'acceptation de cet item :
                                \begin{enumerate}[label*=\arabic*.]
                                    \item Tester de générer la JavaDoc et de la lire.
                                \end{enumerate}
                                \item Post-conditions :
                                \begin{enumerate}[label*=\arabic*.]
                                    \item Les méthodes des contrôleurs seront documentées.
                                \end{enumerate}
                            \end{enumerate}
        \end{enumerate} \\
\hline
    Tests d'acceptation & Les tests seront au niveau visuel. S'il y a un problème d'affichage, on pourra le voir. \\
\hline
    Complexité & 2 \\
\hline
    Effort & 1 \\
\hline
    Commentaires &  \\

\hline
    \rowcolor{Gray}
    \multicolumn{2}{|l|}{5} \\
\hline
    Acteur ou rôle & Programmeur \\
\hline
    Scénario ou story & En tant que programmeur, je
    souhaite acquérir des connaissances sur le fonctionnement des RaspberryPi et des librairies Java permettant son fonctionnement. \\
\hline
    Détail et description &
        \begin{enumerate}[label*=\arabic*.]
            \item \colorbox{Green}{\parbox{13cm}{Chercher une librairie en Java pour le GPIO d'un RaspberryPi.}}
                \begin{enumerate}[label*=\arabic*.]
                                \item Qui et temps :
                                \begin{enumerate}[label*=\arabic*.]
                                    \item (NM)
                                    \item (30 min)
                                \end{enumerate}
                                \item Règles d'affaires :
                                \begin{enumerate}[label*=\arabic*.]
                                    \item Trouver une librairie Java pour le GPIO.
                                \end{enumerate}
                                \item Règles d'affaires alternatives :
                                \begin{enumerate}[label*=\arabic*.]
                                    \item Il n'y a pas d'alternative, car sans librairie, il est impossible de faire fonctionner le RaspberryPi.
                                \end{enumerate}
                                \item Post-conditions :
                                \begin{enumerate}[label*=\arabic*.]
                                    \item Nous possédons une librairie utilisable pour le GPIO du RaspberryPi.
                                \end{enumerate}
                            \end{enumerate}
             \item \colorbox{Green}{\parbox{13cm}{Comprendre la librairie et le GPIO trouvé.}}
                \begin{enumerate}[label*=\arabic*.]
                                \item Qui et temps :
                                \begin{enumerate}[label*=\arabic*.]
                                    \item (NM)
                                    \item (1h45)
                                \end{enumerate}
                                \item Préconditions :
                                \begin{enumerate}[label*=\arabic*.]
                                    \item Avoir trouvé une librairie utilisable et bien documentée.
                                \end{enumerate}
                                \item Règles d'affaires :
                                \begin{enumerate}[label*=\arabic*.]
                                    \item Faire des lectures de la documentation.
                                    \item Regarder des vidéos en ligne sur le sujet.
                                \end{enumerate}
                                \item Règles d'affaires alternatives :
                                \begin{enumerate}[label*=\arabic*.]
                                    \item Il n'y a pas d'alternative puisqu'il faut apprendre ce avec quoi l'on travaille.
                                \end{enumerate}
                                \item Post-conditions :
                                \begin{enumerate}[label*=\arabic*.]
                                    \item On comprend désormais le fonctionnement de la librairie.
                                \end{enumerate}
                            \end{enumerate}                
        \item \colorbox{Green}{\parbox{13cm}{Déterminer s'il existe un simulateur de raspberryPi virtuel.}}
                \begin{enumerate}[label*=\arabic*.]
                                \item Qui et temps :
                                \begin{enumerate}[label*=\arabic*.]
                                    \item (NM)
                                    \item (15 min)
                                \end{enumerate}
                                \item Règles d'affaires :
                                \begin{enumerate}[label*=\arabic*.]
                                    \item Faire des recherches en ligne pour en chercher l'existence.
                                \end{enumerate}
                                \item Règles d'affaires alternatives :
                                \begin{enumerate}[label*=\arabic*.]
                                    \item S'il n'existe pas de simulateur de raspberryPi, nous devrons nous en passer.
                                \end{enumerate}
                                \item Tests d'acceptation de cet item :
                                \begin{enumerate}[label*=\arabic*.]
                                    \item Vérifier que le simulateur fonctionne de façon similaire au vrais.
                                \end{enumerate}
                                \item Post-conditions :
                                \begin{enumerate}[label*=\arabic*.]
                                    \item Nous possédons, ou non, un simulateur de raspberryPi qui fonctionne pour faire nos tests.
                                \end{enumerate}
                            \end{enumerate}
                         \end{enumerate} \\

\hline
    Tests d'acceptation & Vérification pratique des connaissances. \\
\hline
    Complexité & 5 \\
\hline
    Effort & 2 \\
\hline
    Commentaires & \\

\hline
    \rowcolor{Gray}
    \multicolumn{2}{|l|}{6} \\
\hline
    Acteur ou rôle & Programmeur \\
\hline
    Scénario ou story & En tant que programmeur, je
    souhaite pouvoir convertir le courant électrique reçu par le circuit en binaire. \\
\hline
    Détail et description &
        \begin{enumerate}[label*=\arabic*.]
            \item \colorbox{Red}{\parbox{13cm}{Faire un prototype d'application de réception.}}
                \begin{enumerate}[label*=\arabic*.]
                                \item Qui et temps :
                                \begin{enumerate}[label*=\arabic*.]
                                    \item (NM)
                                    \item (5h)
                                \end{enumerate}
                                \item Préconditions :
                                \begin{enumerate}[label*=\arabic*.]
                                    \item Avoir une librairie Java fonctionnelle et la comprendre.
                                \end{enumerate}
                                \item Règles d'affaires :
                                \begin{enumerate}[label*=\arabic*.]
                                    \item Lire le voltage sur un pin du GPIO.
                                    \item Coder le traitement de ce voltage.
                                    \item Reconstruction des octets envoyés.
                                \end{enumerate}
                                \item Règles d'affaires alternatives :
                                \begin{enumerate}[label*=\arabic*.]
                                    \item Il n'y a pas d'alternative, car sans ce code, il est impossible de recréer le fichier envoyé.
                                \end{enumerate}
                                \item Post-conditions :
                                \begin{enumerate}[label*=\arabic*.]
                                    \item Il est possible de décoder l'information reçue.
                                \end{enumerate}
                            \end{enumerate}
            \item \colorbox{Red}{\parbox{13cm}{Programmer l'implémentation finale de la réception.}}
                            \begin{enumerate}[label*=\arabic*.]
                                \item Qui et temps :
                                \begin{enumerate}[label*=\arabic*.]
                                    \item (NM)
                                    \item (2h)
                                \end{enumerate}
                                \item Préconditions :
                                \begin{enumerate}[label*=\arabic*.]
                                    \item Avoir un prototype fonctionnel.
                                \end{enumerate}
                                \item Règles d'affaires :
                                \begin{enumerate}[label*=\arabic*.]
                                    \item Lire le voltage sur un pin du GPIO.
                                    \item Coder le traitement de ce voltage.
                                    \item Reconstruction des octets envoyés.
                                \end{enumerate}
                                \item Règles d'affaires alternatives :
                                \begin{enumerate}[label*=\arabic*.]
                                    \item Il n'y a pas d'alternative, car sans ce code, il est impossible de recréer le fichier envoyé.
                                \end{enumerate}
                                \item Tests d'acceptation de cet item :
                                \begin{enumerate}[label*=\arabic*.]
                                    \item Ce sera un test fonctionnel.
                                    \item Tester unitairement en simulant la réception d'un voltage. 
                                \end{enumerate}
                                \item Post-conditions :
                                \begin{enumerate}[label*=\arabic*.]
                                    \item Le programme est en mesure d'interpréter les voltages reçus e reconstruire un fichier.
                                \end{enumerate}
                            \end{enumerate}
                               \end{enumerate} \\

\hline
    Tests d'acceptation & Les tests seront unitaires en simulant la réception d'un voltage \\
\hline
    Complexité & 8 \\
\hline
    Effort & 4 \\
\hline
    Commentaires & Cette story n'a pas été complètement réalisée, étant donné les problèmes rencontrés avec le Raspberry Pi. Tout ce qui pouvait être fait sans le Raspberry Pi a été fait, donc la story a tout-de-même avancé. Il reste alors à analyser ce que l'on reçoit pour déterminer les bits reçus. Nous n'utiliserons donc pas le Raspberry Pi dans notre projet. Nous utiliserons les librairies offertes par Java pour parvenir à analyser ce que nous recevons. \\
\hline
\end{longtable}
