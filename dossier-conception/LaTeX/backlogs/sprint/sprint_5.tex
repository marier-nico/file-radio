\begin{longtable}{|l|p{0.8\textwidth}|}
\multicolumn{2}{r}{Suite à la page suivante} \\
\endfoot

\multicolumn{2}{r}{} \\
\endlastfoot

\hline
    \rowcolor{Gray}
    \multicolumn{2}{|l|}{1} \\
\hline
    Acteur ou rôle & Développeur  \\
\hline
    Scénario ou story & En tant que développeur, je veux me renseigner sur les FFT, soit les transformations de Fourier rapides. \\
\hline
    Détail et description &
        \begin{enumerate}[label*=\arabic*.]
       \item Recherche d'informations sur Internet sur les transformations de Fourier.
            \begin{enumerate}[label*=\arabic*.]
                    \item Qui et temps :
                    \begin{enumerate}[label*=\arabic*.]
                        \item (TG-H), (GR), (NM) et (CAD)
                        \item (3h)
                    \end{enumerate}
                    \item Préconditions :
                    \begin{enumerate}[label*=\arabic*.]
                    \item Avoir terminer l'émetteur et le récepteur physique.
                    \item Avoir terminer la programmation de l'émission du fichier.
                      \item Avoir terminer la programmation d'une base pour la réception du fichier.
                    \end{enumerate}
                    \item Règles d’affaires :
                    \begin{enumerate}[label*=\arabic*.]
                        \item Recherche sur les transformations de Fourier.
                    \end{enumerate}
                    \item Tests d'acceptation de cet item :
                    \begin{enumerate}[label*=\arabic*.]
                        \item Il n'y en pas vraiment.
                    \end{enumerate}
                    \item Post-conditions :
                    \begin{enumerate}[label*=\arabic*.]
                        \item Comprendre le fonctionnement de base des transformations de Fourier.
                    \end{enumerate}
                \end{enumerate}
                \item Recherche d'informations sur Internet sur les librairies FFT en java.
            \begin{enumerate}[label*=\arabic*.]
                    \item Qui et temps :
                    \begin{enumerate}[label*=\arabic*.]
                        \item (TG-H), (GR), (NM) et (CAD)
                        \item (1h)
                    \end{enumerate}
                    \item Préconditions :
                    \begin{enumerate}[label*=\arabic*.]
                    \item Avoir terminer l'émetteur et le récepteur physique.
                    \item Avoir terminer la programmation de l'émission du fichier.
                      \item Avoir terminer la programmation d'une base pour la réception du fichier.
                    \end{enumerate}
                    \item Règles d’affaires :
                    \begin{enumerate}[label*=\arabic*.]
                        \item Recherche de librairies appropriées au projet.
                    \end{enumerate}
                    \item Tests d'acceptation de cet item :
                    \begin{enumerate}[label*=\arabic*.]
                        \item Il n'y en pas vraiment.
                    \end{enumerate}
                    \item Post-conditions :
                    \begin{enumerate}[label*=\arabic*.]
                        \item Comprendre le fonctionnement des librairies liées aux transformations de Fourier.
                    \end{enumerate}
                \end{enumerate}
                
        \end{enumerate} \\
\hline
    Tests d'acceptation & Appliquer nos apprentissages dans notre programme.\\

\hline
    Complexité & 4 \\
\hline
    Effort & 3 \\
\hline
    Commentaires & \\
    
\hline
    \rowcolor{Gray}
    \multicolumn{2}{|l|}{2} \\
\hline
    Acteur ou rôle & Développeur  \\
\hline
    Scénario ou story & En tant que développeur, je veux appliquer les transformations rapides de Fourier pour filtrer la réception du programme.\\
\hline
    Détail et description &
        \begin{enumerate}[label*=\arabic*.]
            \item   Ajouter les librairies nécessaires au projet. 
            \begin{enumerate}[label*=\arabic*.]
                    \item Qui et temps :
                    \begin{enumerate}[label*=\arabic*.]
                        \item (CAD) et (NM)
                        \item (1h)
                    \end{enumerate}
                    \item Préconditions :
                    \begin{enumerate}[label*=\arabic*.]
                        \item Avoir effectuer la recherche sur les librairies FFT.
                        \item Avoir choisi une librairie convenable au projet.
                    \end{enumerate}
                    \item Règles d’affaires :
                    \begin{enumerate}[label*=\arabic*.]
                        \item Avoir les librairies prêtes à être utilisée.
                    \end{enumerate}
                    \item Règles d’affaires alternative :
                    \begin{enumerate}[label*=\arabic*.]
                        \item Aucune, il est essentiel pour l'accomplissement du projet.
                    \end{enumerate}
                    \item Tests d'acceptation de cet item :
                    \begin{enumerate}[label*=\arabic*.]
                        \item Vérifier qu'il n'y ait aucune erreur d'importation dans le projet.
                    \end{enumerate}
                    \item Post-conditions :
                    \begin{enumerate}[label*=\arabic*.]
                        \item Le programme pourra utiliser la librairie.
                    \end{enumerate}
                \end{enumerate}
            \item  Filtrer l'information reçue par le récepteur à l'aide des transformations rapides de Fourier.
                \begin{enumerate}[label*=\arabic*.]
                    \item Qui et temps :
                    \begin{enumerate}[label*=\arabic*.]
                        \item (CAD) et (NM)
                        \item (5h)
                    \end{enumerate}
                    \item Préconditions : 
                    \begin{enumerate}[label*=\arabic*.]
                        \item Avoir effectuer la recherche nécessaire à une compréhension de base des concepts.
                        \item Avoir importer une librairie FFT en java au projet.
                    \end{enumerate}
                    \item Règles d’affaires :
                    \begin{enumerate}[label*=\arabic*.]
                        \item Filtrer l'information reçue par le récepteur.
                    \end{enumerate}
                    \item Tests d'acceptation de cet item :
                    \begin{enumerate}[label*=\arabic*.]
                        \item Les tests seront faits de manière fonctionnelle.
                    \end{enumerate}
                    \item Post-conditions :
                    \begin{enumerate}[label*=\arabic*.]
                        \item Avoir une information claire permettant de différencier les 1, les 0 et les bruits d'interférences.
                    \end{enumerate}
                \end{enumerate}
        \end{enumerate} \\
\hline
    Tests d'acceptation & Vérifier s'il n'y a pas de perte d'information et si l'information est facilement différentiable. \\

\hline
    Complexité & 5 \\
\hline
    Effort & 7\\
\hline
    Commentaires & \\

\hline
    \rowcolor{Gray}
    \multicolumn{2}{|l|}{3} \\
\hline
    Acteur ou rôle & Développeur \\
\hline
    Scénario ou story & En tant que développeur, je veux pouvoir optimiser les interfaces pour l’envoi et la réception. \\
\hline
    Détail et description &
        \begin{enumerate}[label*=\arabic*.]
            \item Ajouter des thèmes aux interfaces.
                \begin{enumerate}[label*=\arabic*.]
                                \item Qui et temps :
                                \begin{enumerate}[label*=\arabic*.]
                                    \item (CAD) (TGH)
                                    \item (2h)
                                \end{enumerate}
                                \item Règles d'affaires :
                                \begin{enumerate}[label*=\arabic*.]
                                    \item Avoir une multitude de thèmes disponibles aux utilisateurs.
                                \end{enumerate}
                                \item Règles d'affaires alternatives :
                                \begin{enumerate}[label*=\arabic*.]
                                    \item Ne pas avoir une aussi grande diversité de thèmes.
                                \end{enumerate}
                                \item Post-conditions :
                                \begin{enumerate}[label*=\arabic*.]
                                    \item Une vaste variété de thèmes est disponible aux utilisateurs qui peuvent maintenant modifier l'esthétique de l'interface.
                                \end{enumerate}
                            \end{enumerate}
             \item  Ajouter des animations aux transitions entre les différentes fenêtres de l'interface.
                \begin{enumerate}[label*=\arabic*.]
                                \item Qui et temps :
                                \begin{enumerate}[label*=\arabic*.]
                                    \item (CAD)
                                    \item (3h)%À confirmer
                                \end{enumerate}
                                \item Préconditions :
                                \begin{enumerate}[label*=\arabic*.]
                                    \item Connaître la librairie (INSÉRER LE NOM DE LIBRAIRIE) et en comprendre le fonctionnement
                                \end{enumerate}
                                \item Règles d'affaires :
                                \begin{enumerate}[label*=\arabic*.]
                                    \item Rendre la vue plus esthétique et rendre les transitions entre les différentes parties de l'interface plus fluides.
                                \end{enumerate}
                                \item Règles d'affaires alternatives :
                                \begin{enumerate}[label*=\arabic*.]
                                    \item L'interface sera plus rudimentaire et sera moins peaufinée.
                                \end{enumerate}
                                \item Tests d'acceptation de cet item :
                                \begin{enumerate}[label*=\arabic*.]
                                    \item Les tests seront fonctionnels afin de tester que les animations sont fluides lors de situations réelles.
                                \end{enumerate}
                                \item Post-conditions :
                                \begin{enumerate}[label*=\arabic*.]
                                    \item L'interface sera optimisé et semblera plus fluide et fonctionnelle pour l'utilisateur.
                                \end{enumerate}
                            \end{enumerate}
        \end{enumerate} \\
\hline
    Tests d'acceptation & Les tests fonctionnels sont réussis. \\
\hline
    Complexité & 4 \\
\hline
    Effort & 2 \\
\hline
    Commentaires & \\

\hline
    \rowcolor{Gray}
    \multicolumn{2}{|l|}{4} \\
\hline
    Acteur ou rôle & Ingénieur \\
\hline
    Scénario ou story & En tant qu'ingénieur, je souhaite que le circuit électrique soit refaire sur un circuit imprimé. \\
\hline
    Détail et description &
        \begin{enumerate}[label*=\arabic*.]
            \item S'informer auprès de membres du département de technique en génie électrique sur la fabrication de circuit imprimé.
                \begin{enumerate}[label*=\arabic*.]
                                \item Qui et temps :
                                \begin{enumerate}[label*=\arabic*.]
                                    \item (TGH) et (GR)
                                    \item (2h)
                                \end{enumerate}
                                \item Règles d'affaires :
                                \begin{enumerate}[label*=\arabic*.]
                                    \item Connaître le fonctionnement et les critères nécessaires afin de concevoir un circuit imprimé.
                                \end{enumerate}
                                \item Règles d'affaires alternatives :
                                \begin{enumerate}[label*=\arabic*.]
                                    \item Il n'y a pas d'alternative car afin de pouvoir procéder à la fabrication d'un circuit imprimé, nous devons en comprendre les règles.
                                \end{enumerate}
                                \item Post-conditions :
                                \begin{enumerate}[label*=\arabic*.]
                                    \item Nous savons quelles informations nous devons posséder afin de pouvoir fabriquer les circuit imprimés de nos circuits émetteurs et récepteurs.
                                \end{enumerate}
                            \end{enumerate}
             \item  Commander nos circuits imprimés au département de technique en génie électrique.
                \begin{enumerate}[label*=\arabic*.]
                                \item Qui et temps :
                                \begin{enumerate}[label*=\arabic*.]
                                    \item (TGH) et (GR)
                                    \item (1h)
                                \end{enumerate}
                                \item Préconditions :
                                \begin{enumerate}[label*=\arabic*.]
                                    \item Avoir toutes les pré-requis nécessaire à la conception des circuits imprimés.
                                \end{enumerate}
                                \item Règles d'affaires :
                                \begin{enumerate}[label*=\arabic*.]
                                    \item posséder des circuits imprimé pour nos circuits émetteurs et récepteurs.
                                \end{enumerate}
                                \item Règles d'affaires alternatives :
                                \begin{enumerate}[label*=\arabic*.]
                                    \item Continuer à utiliser les anciens circuits assemblés à la main.
                                \end{enumerate}
                                \item Tests d'acceptation de cet item :
                                \begin{enumerate}[label*=\arabic*.]
                                    \item Les tests seront fonctionnels afin de vérifier le fonctionnement du circuit.
                                \end{enumerate}
                                \item Post-conditions :
                                \begin{enumerate}[label*=\arabic*.]
                                    \item Nous posséderons des circuits d'envoi et de réception émettant moins d'interférences et qui enverrons et recevrons un signal plus clair et qui fonctionnera mieux.
                                \end{enumerate}
                            \end{enumerate}
        \end{enumerate} \\
\hline
    Tests d'acceptation & Les tests fonctionnels sont réussis. \\
\hline
    Complexité & 3 \\
\hline
    Effort & 2 \\
\hline
    Commentaires & \\
    
\hline    
\end{longtable}
