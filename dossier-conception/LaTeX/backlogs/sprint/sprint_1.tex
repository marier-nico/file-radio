\begin{longtable}{|l|p{0.8\textwidth}|}
\multicolumn{2}{r}{Suite à la page suivante} \\
\endfoot

\multicolumn{2}{r}{} \\
\endlastfoot

\hline
    \rowcolor{Gray}
    \multicolumn{2}{|l|}{1} \\
\hline
    Acteur ou rôle & Utilisateur \\
\hline
    Scénario ou story & En tant qu'utilisateur, 
      je veux pouvoir sélectionner un fichier
      afin de pouvoir l'envoyer. \\
\hline
    Détail et description &
        \begin{enumerate}[label*=\arabic*.]
            \item Permettre à l’utilisateur d’accéder à sa bibliothèque de fichiers.
            \begin{enumerate}[label*=\arabic*.]
                    \item Qui et temps :
                    \begin{enumerate}[label*=\arabic*.]
                        \item (NM)
                        \item (30 minutes)
                    \end{enumerate}
                    \item Préconditions : 
                    \begin{enumerate}[label*=\arabic*.]
                        \item La présence d'un bouton qui permet à l'utilisateur de procéder à l'action de sélectionner un fichier.
                    \end{enumerate}
                    \item Règles d’affaires :
                    \begin{enumerate}[label*=\arabic*.]
                        \item Doit sélectionner un fichier existant dans l'espace de donné de l'ordinateur utilisé.
                    \end{enumerate}
                    \item Règles d’affaires alternatives :
                    \begin{enumerate}[label*=\arabic*.]
                        \item S'il ne sélectionne pas un fichier existant ou que le fichier sélectionné n'existe plus une fois sélectionné, une alerte est envoyé à l'utilisateur afin qu'il en sélectionne un nouveau.
                    \end{enumerate}
                    \item Tests d'acceptation de cet item :
                    \begin{enumerate}[label*=\arabic*.]
                        \item Tester si le fichier sélectionné est existant dans la base de donnée de l'ordinateur et accessible par le logiciel.
                        \item Vérifier que le contenu du fichier est lisible par le logiciel.
                    \end{enumerate}
                    \item Post-conditions :
                    \begin{enumerate}[label*=\arabic*.]
                        \item L'utilisateur peut sélectionner le fichier qu'il souhaite émettre.
                    \end{enumerate}
                \end{enumerate}
            \item Présenter à l’utilisateur les informations du fichier qu’il s’apprête à envoyer.
                \begin{enumerate}[label*=\arabic*.]
                    \item Qui et temps :
                    \begin{enumerate}[label*=\arabic*.]
                        \item (NM)
                        \item (30 minutes)
                    \end{enumerate}
                    \item Préconditions : 
                    \begin{enumerate}[label*=\arabic*.]
                        \item Le fichier doit être sélectionné.
                    \end{enumerate}
                    \item Règles d’affaires :
                    \begin{enumerate}[label*=\arabic*.]
                        \item Doit pouvoir envoyer l'information à la vue du nom du fichier.
                    \end{enumerate}
                    \item Règles d’affaires alternatives :
                    \begin{enumerate}[label*=\arabic*.]
                        \item Si le nom comportes des caractères non interprétables par l'ordinateur, ce dernier ne les affichera pas.
                    \end{enumerate}
                    \item Tests d'acceptation de cet item :
                    \begin{enumerate}[label*=\arabic*.]
                        \item vérifier chacun des caractère du nom d'un fichier afin de pouvoir afficher tous ceux possibles.
                    \end{enumerate}
                    \item Post-conditions :
                    \begin{enumerate}[label*=\arabic*.]
                        \item L'utilisateur peut voir le nom de son fichier.
                    \end{enumerate}
                \end{enumerate}
        \end{enumerate} \\
\hline
    Tests d'acceptation & Sélectionner un fichier et vérifier qu'il s'affiche correctement. \\
\hline
    Complexité & 1 \\
\hline
    Effort & 1 \\
\hline
    Commentaires &  \\

\hline
    \rowcolor{Gray}
    \multicolumn{2}{|l|}{2} \\
\hline
    Acteur ou rôle & Utilisateur \\
\hline
    Scénario ou story & En tant qu’utilisateur, je
    dois posséder une interface
    visuelle pour l’émission des
    fichiers. \\
\hline
    Détail et description &
        \begin{enumerate}[label*=\arabic*.]
            \item Construire une interface avec
                la disposition souhaitée et les
                fonctions de bases permettant
                d’envoyer un fichier.
                \begin{enumerate}[label*=\arabic*.]
                                \item Qui et temps :
                                \begin{enumerate}[label*=\arabic*.]
                                    \item (CAD)
                                    \item (2h)
                                \end{enumerate}
                                \item Préconditions :
                                \begin{enumerate}[label*=\arabic*.]
                                    \item Avoir préparer un croquis de l'interface.
                                    \item Avoir le logiciel Scene Builder pour la conception de l'interface.
                                \end{enumerate}
                                \item Règles d'affaires :
                                \begin{enumerate}[label*=\arabic*.]
                                    \item Afficher l'interface de l'émetteur.
                                \end{enumerate}
                                \item Règles d'affaires alternatives :
                                \begin{enumerate}[label*=\arabic*.]
                                    \item Il n'y a pas d'alternative, car sans interface, l'utilisateur ne peut rien faire.
                                \end{enumerate}
                                \item Tests d'acceptation de cet item :
                                \begin{enumerate}[label*=\arabic*.]
                                    \item Les tests seront au niveau visuel. S'il y a un problème d'affichage, on pourra le voir.
                                \end{enumerate}
                                \item Post-conditions :
                                \begin{enumerate}[label*=\arabic*.]
                                    \item L'interface devra pouvoir afficher les éléments suivant : Boutons pour la sélections, envoies de fichier.
                                \end{enumerate}
                            \end{enumerate}
             \item Ajouter à l’interface de la couleur et du style.
                \begin{enumerate}[label*=\arabic*.]
                                \item Qui et temps :
                                \begin{enumerate}[label*=\arabic*.]
                                    \item (CAD)
                                    \item (1h)
                                \end{enumerate}
                                \item Préconditions :
                                \begin{enumerate}[label*=\arabic*.]
                                    \item Avoir terminé le fxml.
                                \end{enumerate}
                                \item Règles d'affaires :
                                \begin{enumerate}[label*=\arabic*.]
                                    \item Styliser l'interface en css.
                                \end{enumerate}
                                \item Règles d'affaires alternatives :
                                \begin{enumerate}[label*=\arabic*.]
                                    \item Il n'y a pas d'alternative.
                                \end{enumerate}
                                \item Tests d'acceptation de cet item :
                                \begin{enumerate}[label*=\arabic*.]
                                    \item Les tests seront au niveau visuel. S'il y a un problème d'affichage, on pourra le voir.
                                \end{enumerate}
                                \item Post-conditions :
                                \begin{enumerate}[label*=\arabic*.]
                                    \item L'interface devra être en noir, vert et blanc comme une vue de console.
                                \end{enumerate}
                            \end{enumerate}                
        \end{enumerate} \\
\hline
    Tests d'acceptation & Les tests seront au niveau visuel. S'il y a un problème d'affichage, on pourra le voir. \\
\hline
    Complexité & 1 \\
\hline
    Effort & 1 \\
\hline
    Commentaires &  \\

\hline
    \rowcolor{Gray}
    \multicolumn{2}{|l|}{3} \\
\hline
    Acteur ou rôle & Utilisateur \\
\hline
    Scénario ou story & En tant qu’utilisateur, je veux posséder un circuit
    émetteur afin de pouvoir envoyer les ondes radio contenant l’information
    du fichier sélectionné. \\
\hline
    Détail et description &
        \begin{enumerate}[label*=\arabic*.]
        \item Faire les plans du circuit à produire.
            \begin{enumerate}[label*=\arabic*.]
                    \item Qui et temps :
                    \begin{enumerate}[label*=\arabic*.]
                        \item (TG-H) et (GR)
                        \item (1h)
                    \end{enumerate}
                    \item Règles d’affaires :
                    \begin{enumerate}[label*=\arabic*.]
                        \item Doit être un circuit permettant au courant de circuler et de se rendre à l'antenne.
                        \item le circuit doit pouvoir être alimenté par une prise d'écouteurs d'ordinateur.
                        \item Doit comprendre des composantes permettant d'amplifier le courant transmis par la source d'énergie.
                    \end{enumerate}
                    \item Tests d'acceptation de cet item :
                    \begin{enumerate}[label*=\arabic*.]
                        \item Valider ce dernier en consultant des sources montrant le fonctionnement d'émetteurs radio. 
                    \end{enumerate}
                    \item Post-conditions :
                    \begin{enumerate}[label*=\arabic*.]
                        \item Nous allons avoir un plan de conception et pourrons procéder à la prochaine étape.
                    \end{enumerate}
                \end{enumerate}
            \item Accumuler le matériel nécessaire à la conception du circuit électrique.
            \begin{enumerate}[label*=\arabic*.]
                    \item Qui et temps :
                    \begin{enumerate}[label*=\arabic*.]
                        \item (TG-H) et (GR)
                        \item (30 minutes)
                    \end{enumerate}
                    \item Préconditions :
                    \begin{enumerate}[label*=\arabic*.]
                        \item Avoir un schéma de construction fini et la liste du matériel.
                    \end{enumerate}
                    \item Règles d’affaires :
                    \begin{enumerate}[label*=\arabic*.]
                        \item doit respecter le matériel déterminé dans le schéma de construction.
                        %Yeet moi ca la le dessin thomas chou <3
                    \end{enumerate}
                    \item Tests d'acceptation de cet item :
                    \begin{enumerate}[label*=\arabic*.]
                        \item Vérifier si nous possédons bien chaque pièce nécéssaire à la complétion de l'émetteur.
                    \end{enumerate}
                    \item Post-conditions :
                    \begin{enumerate}[label*=\arabic*.]
                        \item Nous allon pouvoir procéder au montage du circuit émetteur.
                    \end{enumerate}
                \end{enumerate}
            \item Procéder à la conception du circuit de l’émetteur.
                \begin{enumerate}[label*=\arabic*.]
                    \item Qui et temps :
                    \begin{enumerate}[label*=\arabic*.]
                        \item (TG-H) et (GR)
                        \item (4h)
                    \end{enumerate}
                    \item Préconditions : 
                    \begin{enumerate}[label*=\arabic*.]
                        \item Avoir les pièces nécessaires à la conception en main.
                        \item Avoir le plan de conception achevé et vérifié.
                    \end{enumerate}
                    \item Règles d’affaires :
                    \begin{enumerate}[label*=\arabic*.]
                        \item Suivre le schéma de conception et construire le circuit.
                    \end{enumerate}
                    \item Tests d'acceptation de cet item :
                    \begin{enumerate}[label*=\arabic*.]
                        \item Vérifier que le circuit émet bel et bien des onde radio.
                        \item Vérifier que l'information transmise est exacte.
                    \end{enumerate}
                    \item Post-conditions :
                    \begin{enumerate}[label*=\arabic*.]
                        \item L'utilisateur peut émettre des ondes radio à partir de son ordinateur.
                    \end{enumerate}
                \end{enumerate}
        \end{enumerate} \\
\hline
    Tests d'acceptation & Émettre des ondes et en vérifier l'exactitude et bon fonctionnement. \\

\hline
    Complexité & 3 \\
\hline
    Effort & 6 \\
\hline
    Commentaires &  \\

\hline
    \rowcolor{Gray}
    \multicolumn{2}{|l|}{4} \\
\hline
    Acteur ou rôle & Utilisateur \\
\hline
    Scénario ou story & En tant que programmeur, je dois pouvoir convertir l’information contenue dans le fichier sélectionné en code binaire afin que ces dernières soient plus faciles à transporter par ondes radio. \\
\hline
    Détail et description &
        \begin{enumerate}[label*=\arabic*.]
            \item Construire une interface avec la disposition souhaitée et les fonctions de bases permet- tant de décider l’emplacement d’enregistrement du fichier.
                \begin{enumerate}[label*=\arabic*.]
                                \item Qui et temps :
                                \begin{enumerate}[label*=\arabic*.]
                                    \item (NM)
                                    \item (3h)
                                \end{enumerate}
                                \item Préconditions :
                                \begin{enumerate}[label*=\arabic*.]
                                    \item Il faut que la sélection de fichier soit fonctionnelle.
                                \end{enumerate}
                                \item Règles d'affaires :
                                \begin{enumerate}[label*=\arabic*.]
                                    \item Lire le fichier et le convertir le fichier en binaire. 
                                \end{enumerate}
                                \item Tests d'acceptation de cet item :
                                \begin{enumerate}[label*=\arabic*.]
                                    \item Les tests seront de comparer la conversion en binaire avec un fragment du fichier connu.
                                \end{enumerate}
                                \item Post-conditions :
                                \begin{enumerate}[label*=\arabic*.]
                                    \item La conversion devra être complètement fonctionnelle.
                                \end{enumerate}
                            \end{enumerate}
             \item Présenter à l’utilisateur les informations du fichier qu’il s’apprête à envoyer.
                \begin{enumerate}[label*=\arabic*.]
                                \item Qui et temps :
                                \begin{enumerate}[label*=\arabic*.]
                                    \item (NM)
                                    \item (1h)
                                \end{enumerate}
                                \item Préconditions :
                                \begin{enumerate}[label*=\arabic*.]
                                    \item Pouvoir sélectionner un fichier.
                                \end{enumerate}
                                \item Règles d'affaires :
                                \begin{enumerate}[label*=\arabic*.]
                                    \item Extraction de donnés de type String.
                                \end{enumerate}
                                \item Règles d'affaires alternatives :
                                \begin{enumerate}[label*=\arabic*.]
                                    \item Prendra une String avec comme valeur : Nom inconnu.
                                \end{enumerate}
                                \item Tests d'acceptation de cet item :
                                \begin{enumerate}[label*=\arabic*.]
                                    \item Les tests seront au niveau visuel, car nous pourrons voir si l'affichage s'effectue.
                                \end{enumerate}
                                \item Post-conditions :
                                \begin{enumerate}[label*=\arabic*.]
                                    \item L'information du fichier sera présenté à l'écran.
                                \end{enumerate}
                            \end{enumerate}                
        \end{enumerate} \\
\hline
    Tests d'acceptation & Les tests seront de comparer la conversion en binaire avec un fragment du fichier connu. \\
\hline
    Complexité & 1 \\
\hline
    Effort & 2 \\
\hline
    Commentaires &  \\
\hline
\end{longtable}