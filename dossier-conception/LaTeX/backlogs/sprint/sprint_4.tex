\begin{longtable}{|l|p{0.8\textwidth}|}
\multicolumn{2}{r}{Suite à la page suivante} \\
\endfoot

\multicolumn{2}{r}{} \\
\endlastfoot

\hline
    \rowcolor{Gray}
    \multicolumn{2}{|l|}{1} \\
\hline
    Acteur ou rôle & Développeur  \\
\hline
    Scénario ou story & En tant que développeur, je veux me renseigner sur les API et les librairies à disposition pour faire la réception du son. \\
\hline
    Détail et description &
        \begin{enumerate}[label*=\arabic*.]
       \item \colorbox{Green}{\parbox{13cm}{Recherche d'informations sur Internet et prendre le temps de lire les librairies.}}
            \begin{enumerate}[label*=\arabic*.]
                    \item Qui et temps :
                    \begin{enumerate}[label*=\arabic*.]
                        \item (TG-H), (GR), (NM) et (CAD)
                        \item (2h)
                    \end{enumerate}
                    \item Préconditions :
                    \begin{enumerate}[label*=\arabic*.]
                    \item Avoir terminer l'émetteur et le récepteur physique.
                     \item Avoir terminer la programmation de l'émission du fichier.
                    \end{enumerate}
                    \item Règles d’affaires :
                    \begin{enumerate}[label*=\arabic*.]
                        \item Recherche sur les API Java son.
                        \item Rencontre sur les librairies Java son.
                    \end{enumerate}
                    \item Tests d'acceptation de cet item :
                    \begin{enumerate}[label*=\arabic*.]
                        \item Il n'y en pas vraiment.
                    \end{enumerate}
                    \item Post-conditions :
                    \begin{enumerate}[label*=\arabic*.]
                        \item Comprendre le fonctionnement de réception sonore avec java.
                    \end{enumerate}
                \end{enumerate}
        \end{enumerate} \\
\hline
    Tests d'acceptation & Appliquer nos apprentissages dans notre programme.\\

\hline
    Complexité & 2 \\
\hline
    Effort & 3 \\
\hline
    Commentaires & \\
    
\hline
    \rowcolor{Gray}
    \multicolumn{2}{|l|}{2} \\
\hline
    Acteur ou rôle & Développeur  \\
\hline
    Scénario ou story & En tant que développeur, je veux appliquer des modifications à la vue correspondant à la réception de fichier. \\
\hline
    Détail et description &
        \begin{enumerate}[label*=\arabic*.]
            \item \colorbox{Green}{\parbox{13cm}{Lier la vue réception aux classes métiers permettant de gérer la réception}}
            \begin{enumerate}[label*=\arabic*.]
                    \item Qui et temps :
                    \begin{enumerate}[label*=\arabic*.]
                        \item (CAD) et (TG-H)
                        \item (3h)
                    \end{enumerate}
                    \item Préconditions :
                    \begin{enumerate}[label*=\arabic*.]
                        \item Avoir terminer la vue réception.
                        \item Avoir terminer les classes métiers.
                    \end{enumerate}
                    \item Règles d’affaires :
                    \begin{enumerate}[label*=\arabic*.]
                        \item Avoir une vue permettant la réception et l'enregistrement de fichier.
                    \end{enumerate}
                    \item Règles d’affaires alternative :
                    \begin{enumerate}[label*=\arabic*.]
                        \item Aucune, il est essentiel pour l'accomplissement du projet.
                    \end{enumerate}
                    \item Tests d'acceptation de cet item :
                    \begin{enumerate}[label*=\arabic*.]
                        \item Vérifier si on est capable de réceptionner le message envoyé en comparant ce que l'on enregistre et ce que l'on a envoyé.
                    \end{enumerate}
                    \item Post-conditions :
                    \begin{enumerate}[label*=\arabic*.]
                        \item Le programme pourra enregistrer les fichiers envoyés.
                    \end{enumerate}
                \end{enumerate}
            \item \colorbox{Green}{\parbox{13cm}{Optimiser la configuration de l'interface récepteur.}}
                \begin{enumerate}[label*=\arabic*.]
                    \item Qui et temps :
                    \begin{enumerate}[label*=\arabic*.]
                        \item (CAD) et (TG-H)
                        \item (4h)
                    \end{enumerate}
                    \item Préconditions : 
                    \begin{enumerate}[label*=\arabic*.]
                        \item Avoir terminer la vue réception.
                    \end{enumerate}
                    \item Règles d’affaires :
                    \begin{enumerate}[label*=\arabic*.]
                        \item Avoir une vue avec les outils nécessaires pour visualiser et gérer la réception.
                    \end{enumerate}
                    \item Tests d'acceptation de cet item :
                    \begin{enumerate}[label*=\arabic*.]
                        \item Le tests se feront visuellement.
                    \end{enumerate}
                    \item Post-conditions :
                    \begin{enumerate}[label*=\arabic*.]
                        \item Avoir une vue avec les éléments nécessaires pour visualiser la réception.
                    \end{enumerate}
                \end{enumerate}
        \end{enumerate} \\
\hline
    Tests d'acceptation & Vérifier si on est capable de réceptionner le message envoyé en comparant ce que l'on enregistre et ce que l'on a envoyé.\\

\hline
    Complexité & 3 \\
\hline
    Effort & 5 \\
\hline
    Commentaires & \\

\hline
    \rowcolor{Gray}
    \multicolumn{2}{|l|}{3} \\
\hline
    Acteur ou rôle & Développeur \\
\hline
    Scénario ou story & En tant que développeur, je veux pouvoir analyser les données reçues par radio afin de déterminer si on reçoit un 1 ou un 0. \\
\hline
    Détail et description &
        \begin{enumerate}[label*=\arabic*.]
            \item \colorbox{Green}{\parbox{13cm}{Accéder aux données entrantes par la prise du microphone d'un ordinateur.}}
                \begin{enumerate}[label*=\arabic*.]
                                \item Qui et temps :
                                \begin{enumerate}[label*=\arabic*.]
                                    \item (NM) et (GR)
                                    \item (2h)
                                \end{enumerate}
                                \item Préconditions :
                                \begin{enumerate}[label*=\arabic*.]
                                    \item Avoir fait la recherche à ce sujet.
                                \end{enumerate}
                                \item Règles d'affaires :
                                \begin{enumerate}[label*=\arabic*.]
                                    \item Pouvoir accéder aux données sonores du microphone.
                                \end{enumerate}
                                \item Règles d'affaires alternatives :
                                \begin{enumerate}[label*=\arabic*.]
                                    \item Il n'y a pas d'alternative, cette étape est absolument nécessaire, car il est impossible de reconstruire le fichier si nous n'avons pas accès à ces informations.
                                \end{enumerate}
                                \item Tests d'acceptation de cet item :
                                \begin{enumerate}[label*=\arabic*.]
                                    \item Les tests seront unitaires.
                                \end{enumerate}
                                \item Post-conditions :
                                \begin{enumerate}[label*=\arabic*.]
                                    \item Nous avons maintenant accès aux informations provenant de la prise du micro et nous pouvons alors manipuler ces informations.
                                \end{enumerate}
                            \end{enumerate}
            \item \colorbox{Green}{\parbox{13cm}{Analyser les données reçues par la prise du microphone et déterminer si nous recevons un 1 ou un 0.}}
                \begin{enumerate}[label*=\arabic*.]
                                \item Qui et temps :
                                \begin{enumerate}[label*=\arabic*.]
                                    \item (NM) et (GR)
                                    \item (4h)
                                \end{enumerate}
                                \item Préconditions :
                                \begin{enumerate}[label*=\arabic*.]
                                    \item Avoir accès aux informations provenant de la prise du microphone.
                                \end{enumerate}
                                \item Règles d'affaires :
                                \begin{enumerate}[label*=\arabic*.]
                                    \item Déterminer si les données représentent un 1 ou un 0.
                                \end{enumerate}
                                \item Règles d'affaires alternatives :
                                \begin{enumerate}[label*=\arabic*.]
                                    \item Il n'y a pas d'alternative, cette étape est absolument nécessaire, car il est impossible de reconstruire le fichier si nous n'avons pas accès à ces informations.
                                \end{enumerate}
                                \item Tests d'acceptation de cet item :
                                \begin{enumerate}[label*=\arabic*.]
                                    \item Les tests seront fonctionnels, car on ne peut pas simuler la réception de données du microphone.
                                \end{enumerate}
                                \item Post-conditions :
                                \begin{enumerate}[label*=\arabic*.]
                                    \item Il est possible de déterminer si nous recevons un 1 ou un 0.
                                \end{enumerate}
                            \end{enumerate}
        \end{enumerate} \\
\hline
    Tests d'acceptation & Les tests unitaires passent et les tests fonctionnels sont réussis. \\
\hline
    Complexité & 7 \\
\hline
    Effort & 4 \\
\hline
    Commentaires & \\
    
\hline
\end{longtable}
