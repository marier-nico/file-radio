\begin{longtable}{|l|p{0.8\textwidth}|}
\multicolumn{2}{r}{Suite à la page suivante} \\
\endfoot

\multicolumn{2}{r}{} \\
\endlastfoot

\hline
    \rowcolor{Gray}
    \multicolumn{2}{|l|}{1} \\
\hline
    Acteur ou rôle & Utilisateur \\
\hline
    Scénario ou story & En tant qu’utilisateur, je veux posséder un circuit
    récepteur afin de pouvoir capter les ondes radio produites par l’émetteur. \\
\hline
    Détail et description &
        \begin{enumerate}[label*=\arabic*.]
        \item\colorbox{Green}{\parbox{13cm}{ Faire les plans du circuit à produire.}}
            \begin{enumerate}[label*=\arabic*.]
                    \item Qui et temps :
                    \begin{enumerate}[label*=\arabic*.]
                        \item (TG-H) et (GR)
                        \item (1h)
                    \end{enumerate}
                    \item Règles d’affaires :
                    \begin{enumerate}[label*=\arabic*.]
                        \item Doit être un circuit permettant à l'antenne de réception de transmettre le courant créé par les ondes radio de se rendre à l'ordinateur.
                        \item le circuit doit pouvoir être relié à l'ordinateur par la prise d'écouteurs.
                    \end{enumerate}
                    \item Tests d'acceptation de cet item :
                    \begin{enumerate}[label*=\arabic*.]
                        \item Valider ce dernier en consultant des sources montrant le fonctionnement de récepteurs d'ondes radio. 
                    \end{enumerate}
                    \item Post-conditions :
                    \begin{enumerate}[label*=\arabic*.]
                        \item Nous allons avoir un plan de conception et pourrons procéder à la prochaine étape.
                    \end{enumerate}
                \end{enumerate}
            \item \colorbox{Green}{\parbox{13cm}{ Accumuler le matériel nécessaire à la conception du circuit électrique.}}
            \begin{enumerate}[label*=\arabic*.]
                    \item Qui et temps :
                    \begin{enumerate}[label*=\arabic*.]
                        \item (TG-H) et (GR)
                        \item (30 minutes)
                    \end{enumerate}
                    \item Préconditions :
                    \begin{enumerate}[label*=\arabic*.]
                        \item Avoir un schéma de construction fini et la liste du matériel.
                    \end{enumerate}
                    \item Règles d’affaires :
                    \begin{enumerate}[label*=\arabic*.]
                        \item doit respecter le matériel déterminé dans le schéma de construction.
                        %Yeet moi ca la le dessin thomas chou <3
                    \end{enumerate}
                    \item Tests d'acceptation de cet item :
                    \begin{enumerate}[label*=\arabic*.]
                        \item Vérifier si nous possédons bien chaque pièce nécessaire à la complétion du récepteur.
                    \end{enumerate}
                    \item Post-conditions :
                    \begin{enumerate}[label*=\arabic*.]
                        \item Nous allons pouvoir procéder au montage du circuit récepteur.
                    \end{enumerate}
                \end{enumerate}
            \item \colorbox{Green}{\parbox{13cm}{ Procéder à la conception du circuit du récepteur.}}
                \begin{enumerate}[label*=\arabic*.]
                    \item Qui et temps :
                    \begin{enumerate}[label*=\arabic*.]
                        \item (TG-H) et (GR)
                        \item (4h)
                    \end{enumerate}
                    \item Préconditions : 
                    \begin{enumerate}[label*=\arabic*.]
                        \item Avoir les pièces nécessaires à la conception en main.
                        \item Avoir le plan de conception achevé et vérifié.
                    \end{enumerate}
                    \item Règles d’affaires :
                    \begin{enumerate}[label*=\arabic*.]
                        \item Suivre le schéma de conception et construire le circuit.
                    \end{enumerate}
                    \item Tests d'acceptation de cet item :
                    \begin{enumerate}[label*=\arabic*.]
                        \item Vérifier que le circuit reçoit bel et bien des ondes radio.
                        \item Vérifier que l'information transmise à l'ordinateur est bien la même que celle envoyée.
                    \end{enumerate}
                    \item Post-conditions :
                    \begin{enumerate}[label*=\arabic*.]
                        \item L'utilisateur peut capter des ondes radio et les enregistrer sur son ordinateur.
                    \end{enumerate}
                \end{enumerate}
        \end{enumerate} \\
\hline
    Tests d'acceptation & Essayer de capter des ondes et en vérifier l'exactitude. \\

\hline
    Complexité & 3 \\
\hline
    Effort & 6 \\
\hline
    Commentaires & \\

\hline
    \rowcolor{Gray}
    \multicolumn{2}{|l|}{2} \\
\hline
    Acteur ou rôle & Utilisateur \\
\hline
    Scénario ou story & En tant qu’utilisateur, je dois posséder une interface visuelle pour la réception des fichiers. \\
\hline
    Détail et description &
        \begin{enumerate}[label*=\arabic*.]
            \item \colorbox{Green}{\parbox{13cm}{Construire une interface avec la disposition souhaitée et les fonctions de bases permettant de décider l’emplacement d’enregistrement du fichier.}}
                \begin{enumerate}[label*=\arabic*.]
                                \item Qui et temps :
                                \begin{enumerate}[label*=\arabic*.]
                                    \item (CAD)
                                    \item (2h)
                                \end{enumerate}
                                \item Préconditions :
                                \begin{enumerate}[label*=\arabic*.]
                                    \item Avoir préparer un croquis de l'interface.
                                    \item Avoir le logiciel Scene Builder pour la conception de l'interface.
                                \end{enumerate}
                                \item Règles d'affaires :
                                \begin{enumerate}[label*=\arabic*.]
                                    \item Afficher l'interface du récepteur.
                                \end{enumerate}
                                \item Règles d'affaires alternatives :
                                \begin{enumerate}[label*=\arabic*.]
                                    \item Il n'y a pas d'alternative, car sans interface, l'utilisateur ne peut rien faire.
                                \end{enumerate}
                                \item Tests d'acceptation de cet item :
                                \begin{enumerate}[label*=\arabic*.]
                                    \item Les tests seront au niveau visuel. S'il y a un problème d'affichage, on pourra le voir.
                                \end{enumerate}
                                \item Post-conditions :
                                \begin{enumerate}[label*=\arabic*.]
                                    \item L'interface devra pouvoir afficher les éléments suivant : Boutons pour la sélections et pour l'envoi de fichier.
                                \end{enumerate}
                            \end{enumerate}
             \item \colorbox{Green}{\parbox{13cm}{ Ajouter à l’interface de la couleur et du style.}}
                \begin{enumerate}[label*=\arabic*.]
                                \item Qui et temps :
                                \begin{enumerate}[label*=\arabic*.]
                                    \item (CAD)
                                    \item (1h)
                                \end{enumerate}
                                \item Préconditions :
                                \begin{enumerate}[label*=\arabic*.]
                                    \item Avoir terminé le fxml.
                                \end{enumerate}
                                \item Règles d'affaires :
                                \begin{enumerate}[label*=\arabic*.]
                                    \item Styliser l'interface en css.
                                \end{enumerate}
                                \item Règles d'affaires alternatives :
                                \begin{enumerate}[label*=\arabic*.]
                                    \item Il n'y a pas d'alternative.
                                \end{enumerate}
                                \item Tests d'acceptation de cet item :
                                \begin{enumerate}[label*=\arabic*.]
                                    \item Les tests seront au niveau visuel. S'il y a un problème d'affichage, on pourra le voir.
                                \end{enumerate}
                                \item Post-conditions :
                                \begin{enumerate}[label*=\arabic*.]
                                    \item L'interface devra pouvoir nous diriger correctement sur la bonne vue selon la sélection de l'utilisateur.
                                \end{enumerate}
                            \end{enumerate}                
        \end{enumerate} \\
\hline
    Tests d'acceptation & Les tests seront au niveau visuel. S'il y a un problème d'affichage, on pourra le voir. \\
\hline
    Complexité & 1 \\
\hline
    Effort & 1 \\
\hline
    Commentaires &  \\

\hline
    \rowcolor{Gray}
    \multicolumn{2}{|l|}{3} \\
\hline
    Acteur ou rôle & Utilisateur \\
\hline
    Scénario ou story & En tant qu’utilisateur, je
    dois posséder une interface
    visuelle pour le menu permettant d'accéder aux différentes vues. \\
\hline
    Détail et description &
        \begin{enumerate}[label*=\arabic*.]
            \item \colorbox{Green}{\parbox{13cm}{ Concevoir une vue pour le menu.}}
                \begin{enumerate}[label*=\arabic*.]
                                \item Qui et temps :
                                \begin{enumerate}[label*=\arabic*.]
                                    \item (CAD)
                                    \item (1h)
                                \end{enumerate}
                                \item Préconditions :
                                \begin{enumerate}[label*=\arabic*.]
                                    \item Avoir préparer un croquis de l'interface.
                                    \item Avoir le logiciel Scene Builder pour la conception de l'interface.
                                \end{enumerate}
                                \item Règles d'affaires :
                                \begin{enumerate}[label*=\arabic*.]
                                    \item Afficher l'interface du menu.
                                \end{enumerate}
                                \item Règles d'affaires alternatives :
                                \begin{enumerate}[label*=\arabic*.]
                                    \item Il n'y a pas d'alternative, car sans interface, l'utilisateur ne peut rien faire.
                                \end{enumerate}
                                \item Tests d'acceptation de cet item :
                                \begin{enumerate}[label*=\arabic*.]
                                    \item Les tests seront au niveau visuel. S'il y a un problème d'affichage, on pourra le voir.
                                \end{enumerate}
                                \item Post-conditions :
                                \begin{enumerate}[label*=\arabic*.]
                                    \item L'interface devra pouvoir afficher deux boutons pour les deux autres vues.
                                \end{enumerate}
                            \end{enumerate}
             \item \colorbox{Green}{\parbox{13cm}{ Relier la vue récepteur et la vue émetteur à la vue menu.}}
                \begin{enumerate}[label*=\arabic*.]
                                \item Qui et temps :
                                \begin{enumerate}[label*=\arabic*.]
                                    \item (CAD)
                                    \item (1h)
                                \end{enumerate}
                                \item Préconditions :
                                \begin{enumerate}[label*=\arabic*.]
                                    \item Avoir terminé le fxml des autres vues.
                                \end{enumerate}
                                \item Règles d'affaires :
                                \begin{enumerate}[label*=\arabic*.]
                                    \item L'interaction se fera via le contrôleur en Java.
                                \end{enumerate}
                                \item Règles d'affaires alternatives :
                                \begin{enumerate}[label*=\arabic*.]
                                    \item Il n'y a pas d'alternative.
                                \end{enumerate}
                                \item Tests d'acceptation de cet item :
                                \begin{enumerate}[label*=\arabic*.]
                                    \item Les tests seront au niveau visuel. S'il y a un problème d'affichage, on pourra le voir.
                                \end{enumerate}
                                \item Post-conditions :
                                \begin{enumerate}[label*=\arabic*.]
                                    \item L'interface devra être en noir, en vert et en blanc comme une vue de console.
                                \end{enumerate}
                            \end{enumerate}                
        \end{enumerate} \\
\hline
    Tests d'acceptation & Les tests seront au niveau visuel. S'il y a un problème d'affichage, on pourra le voir. \\
\hline
    Complexité & 1 \\
\hline
    Effort & 1 \\
\hline
    Commentaires &  \\

\hline
    \rowcolor{Gray}
    \multicolumn{2}{|l|}{4} \\
\hline
    Acteur ou rôle & Programmeur \\
\hline
    Scénario ou story & En tant que programmeur, je veux convertir le code binaire en son afin de pouvoir prochainement le transférer par la prise audio. \\
\hline
    Détail et description & 
        \begin{enumerate}[label*=\arabic*.]
        \item \colorbox{Green}{\parbox{13cm}{ Convertir le code binaire en son.}}
            \begin{enumerate}[label*=\arabic*.]
                    \item Qui et temps :
                    \begin{enumerate}[label*=\arabic*.]
                        \item (NM)
                        \item (2h)
                    \end{enumerate}
                    \item Règles d’affaires :
                    \begin{enumerate}[label*=\arabic*.]
                        \item Utiliser l'API MidiChannel de Java pour synthétiser le binaire en son.
                    \end{enumerate}
                    \item Tests d'acceptation de cet item :
                    \begin{enumerate}[label*=\arabic*.]
                        \item Valider ce dernier en diffusant ce son par la sortie audio de l'ordinateur (caisse de son).
                    \end{enumerate}
                    \item Post-conditions :
                    \begin{enumerate}[label*=\arabic*.]
                        \item Le son obtenu sera le résultat du fichier en binaire une fois converti.
                    \end{enumerate}
                \end{enumerate}
        \end{enumerate} \\
\hline
    Tests d'acceptation & Valider ce dernier en diffusant se son par la sortie audio de l'ordinateur (caisse de son). \\
\hline
    Complexité & 2 \\
\hline
    Effort & 2 \\
\hline
    Commentaires &  \\

\hline
    \rowcolor{Gray}
    \multicolumn{2}{|l|}{5} \\
\hline
    Acteur ou rôle & Utilisateur \\
\hline
    Scénario ou story & En tant qu'utilisateur, 
      je veux pouvoir émettre le son créé par la conversion de mon ficher afin de pouvoir l'envoyer à un récepteur. \\
\hline
    Détail et description &
        \begin{enumerate}[label*=\arabic*.]
            \item \colorbox{Green}{\parbox{13cm}{ Faire jouer le son contenant l'information.}}
            \begin{enumerate}[label*=\arabic*.]
                    \item Qui et temps :
                    \begin{enumerate}[label*=\arabic*.]
                        \item (NM)
                        \item (30 minutes)
                    \end{enumerate}
                    \item Préconditions : 
                    \begin{enumerate}[label*=\arabic*.]
                        \item Posséder le son créé de la conversion du fichier.
                    \end{enumerate}
                    \item Règles d’affaires :
                    \begin{enumerate}[label*=\arabic*.]
                        \item Sélectionner le son créé de la conversion.
                        \item Faire jouer le son.
                    \end{enumerate}
                    \item Règles d’affaires alternatives :
                    \begin{enumerate}[label*=\arabic*.]
                        \item Si aucun son n'est créé, ne rien émettre.
                    \end{enumerate}
                    \item Tests d'acceptation de cet item :
                    \begin{enumerate}[label*=\arabic*.]
                        \item Faire un test sonore.
                    \end{enumerate}
                    \item Post-conditions :
                    \begin{enumerate}[label*=\arabic*.]
                        \item L'utilisateur peut jouer le son.
                    \end{enumerate}
                \end{enumerate}
        \end{enumerate} \\
\hline
    Tests d'acceptation & Faire un test sonore pour voir si un son est créé. \\
\hline
    Complexité & 1 \\
\hline
    Effort & 1 \\
\hline
    Commentaires &  \\
\hline
\end{longtable}
